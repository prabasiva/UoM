\section{Dimensionality}

The main drawback of the Wigner Ville distribution is cross term interference and the cross term oscillates and is localized. 

Time Frequency Distribution Series (TFDS) was introduced by Chen and Qian \cite{pchen} as the decomposition of the Wigner Ville distribution via the orthogonal like Gabor expansion. Let me walk through each step to attain the time frequency distribution series. 

Let $g(t)$ be a normalized Gaussian function which is defined as follows. 

\begin{equation}
WVD_g(t,\omega) = 2 e^{-(\frac{t^2}{\sigma^2}+\sigma^2\omega^2)}
\end{equation}

\subsection{Correlation Dimension}
The correlation dimension provides a tool to quantify self-similarity. A larger correlation dimension corresponds to a larger degree of complexity and less self-similarity. The most frequently used procedure to estimate the correlation dimension was introduced by Grassberger and Procaccia (1983). They defined the correlation sum for a collection of points $X_i$ $(i=1,2,3,..N)$ in some phase space to be the fraction of all possible pairs of points which are closer than a given distance $\epsilon$ in a particular norm:

\begin{equation} \label{cd:1}
C(m,\epsilon) = \frac{2}{(N-m)(N-m-1)} \sum_{i=m}^{N}\sum_{j=i+1}^{N} \Theta (\epsilon - \lVert X_i - X_j \rVert )
\end{equation}

Where $\Theta$ is Heaviside step function, $\Theta(x) = 0$ if $x \le 0 $ and $\Theta(x) = 1$ for $x>0$.
Thus equation \ref{cd:1} counts the pairs $(X_i,X_j)$ whose distance is smaller than $\epsilon$.  When ${N\to\infty}$, for small values of $\epsilon$, C follows a power law;

\begin{equation} \label{cd:2}
C(\epsilon) \propto \epsilon^{D_c}
\end{equation}

Where $D_c$ is correlation dimension. Therefore, $D_c$ is defined as 
\begin{equation} \label{cd:3}
D_c = \lim_{\epsilon\to 0} \lim_{N\to\infty} \frac{\partial ln C(\epsilon)}{\partial ln(\epsilon)}
\end{equation}

Correlation dimension is estimated by computing the slope of the straight line by using least sqare fit in a plot of $ln C(\epsilon)$ vs $ln(\epsilon)$. 
